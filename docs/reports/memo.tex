\documentclass[aspectratio=169, dvipdfmx, hyperref={bookmarks=true}]{beamer}

%%%%%% config %%%%%%
\usetheme{Madrid}
\title[Beamer]{Beamerのテスト}
\subtitle{サブタイトル}
\author[Name]{氏名}
\institute[Affiliation]{所属}
\date{\today}

%%%%%% packages %%%%%%
\usepackage{pxjahyper}
\usepackage{bxdpx-beamer}
\usepackage{minijs}

%%%%%% babel %%%%%%
\uselanguage{japanese}
\languagepath{japanese}
\deftranslation[to=japanese]{Theorem}{定理}
\deftranslation[to=japanese]{Lemma}{補題}
\deftranslation[to=japanese]{Example}{例}
\deftranslation[to=japanese]{Examples}{例}
\deftranslation[to=japanese]{Definition}{定義}
\deftranslation[to=japanese]{Definitions}{定義}
\deftranslation[to=japanese]{Problem}{問題}
\deftranslation[to=japanese]{Solution}{解}
\deftranslation[to=japanese]{Fact}{事実}
\deftranslation[to=japanese]{Proof}{証明}

%%%%%% fonts %%%%%%
\usefonttheme{professionalfonts}
\usepackage{txfonts}  % TXフォント
\renewcommand{\kanjifamilydefault}{\gtdefault}  % 和文用
\renewcommand{\familydefault}{\sfdefault}  % 英文をサンセリフ体に
%\usepackage[T1]{fontenc}%8bit フォント
%\usepackage{textcomp}%欧文フォントの追加
%\usepackage[utf8]{inputenc}%文字コードをUTF-8
%\usepackage{otf}%otfパッケージ
%\usepackage{lxfonts}%数式・英文ローマン体を Lxfont にする
%\usepackage{bm}%数式太字

\begin{document}

\begin{frame}
\maketitle
\end{frame}

\section{節}

\begin{frame}{目次}
  \tableofcontents
\end{frame}

\begin{frame}
\frametitle{1枚目のスライドのタイトル}
テスト テスト テスト テスト テスト テスト テスト テスト

テスト テスト テスト テスト テスト テスト
\end{frame}

\begin{frame}
\frametitle{2枚目のスライドのタイトル}
  % 通常のブロック
  \begin{block}{block}
      simple block
    \end{block}
    % 注意のブロック
    \begin{alertblock}{alertblock}
      \begin{itemize}
        \item これは注意です
        \item とても重要です
      \end{itemize}
      \begin{itemize}
        \item これは注意です
        \item とても重要です
      \end{itemize}
    \end{alertblock}
    % Example ブロック
    \begin{exampleblock}{exampleblock}
      exampleblock
    \end{exampleblock}
\end{frame}

\begin{frame}
  \frametitle{3枚目のスライドのタイトル}
  \begin{theorem}[定理の名前]
    これは定理です。
    成立して嬉しいね。
    \begin{equation}
      a^2 + b^2 = c^2
    \end{equation}
  \end{theorem}
\end{frame}

\end{document}
